%
% Curriculum Vitae is the personal CV of Giulio Biagini.
%
% Copyright (C) 2017 Giulio Biagini - giulio.biagini90@gmail.com
%
% This program is free software: you can redistribute it and/or modify
% it under the terms of the GNU General Public License as published by
% the Free Software Foundation, either version 3 of the License, or
% (at your option) any later version.
%
% This program is distributed in the hope that it will be useful,
% but WITHOUT ANY WARRANTY; without even the implied warranty of
% MERCHANTABILITY or FITNESS FOR A PARTICULAR PURPOSE.  See the
% GNU General Public License for more details.
%
% You should have received a copy of the GNU General Public License
% along with this program.  If not, see <http://www.gnu.org/licenses/>.
%



\clearpage

\recipient{Spett. Dr. Giuseppe Basile}{
	SaGa Coffee S.p.A.\\
	Località Casona 1066\\
	40041, Gaggio Montano, Bologna
}

\date{\today}

\opening{Oggetto: Autocandidatura}

\makelettertitle

Sono un ragazzo laureato in Informatica fortemente interessato a lavorare nella
Vostra azienda, multinazionale di fama internazionale e realtà legata al
territorio nel quale sono nato e vissuto.\newline
\newline
Ho seguito il Corso di Laurea Triennale in Informatica all'Università di Bologna
dove mi sono laureato con ottimi voti e sto attualmente frequentando il Corso di
Laurea Magistrale in Informatica. Il Curriculum scelto è quello di Sistemi e
Reti. Sono prossimo al conseguimento della Laurea, data fissata per Marzo
2018.\newline
\newline
Durante il percorso di studi affrontato, fra le altre cose, mi sono più volte
trovato a dover lavorare con teconologie e linguaggi di ``basso livello'', a
partire dallo svolgimento di un progetto circa l'implementazione di un algoritmo
in \textsc{Assembly 8086}, alla realizzazione di numerosi progetti in
\textsc{ANSI C}, sino allo svolgimento del progetto di Laurea, che mi ha portato
a lavorare in \textsc{Wiring}, un linguaggio derivato dal \textsc{C} per la
programmazione di schede \textsc{Arduino}.\newline
\newline
Ho comandato il movimento di un robot da smartphone tramite interfaccia di
comunicazione Bluetooth, creando da zero l'applicazione Android, programmando
interamente la scheda Arduino alla quale sono collegati sia i motori che
permettono il movimento del robot che un chip Bluetooth per lo scambio dei dati,
così come ho interamente progettato e realizzato sia il framework per la
comunicazione applicazione/robot che quello relativo al movimento
scheda/motori.\newline
\newline
È possibile trovare il testo della Tesi di Laurea che descrive l'intero progetto
al seguente URL:\newline
\small{\url{http://giulio.biagini.web.cs.unibo.it/biagini_giulio_tesi_triennale_informatica.pdf}}\newline
\newline
Utilizzo quotidianamente la distribuzione Debian basata su kernel Linux, sono un
amante della programmazione Object-Oriented ed appassionato di Ingegneria del
Software. I numerosi progetti affrontati durante la carriera universitaria mi
hanno reso un buon membro per il lavoro all'interno di team, così come la
realizzazione di progetti personali hanno permesso l'affinamento delle mie
capacità di lavorare in autonomia.\newline
\newline
RingraziandoVi per l'attenzione, Vi comunico la mia più completa disponibilità
per lo svolgimento di un eventuale colloquio informativo.

\closing{Cordiali saluti,}

\enclosure[Attached]{curriculum vit\ae}

\makeletterclosing

\newpage
