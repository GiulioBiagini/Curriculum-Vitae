%
% Curriculum Vitae is the personal CV of Giulio Biagini.
%
% Copyright (C) 2017 Giulio Biagini - giulio.biagini90@gmail.com
%
% This program is free software: you can redistribute it and/or modify
% it under the terms of the GNU General Public License as published by
% the Free Software Foundation, either version 3 of the License, or
% (at your option) any later version.
%
% This program is distributed in the hope that it will be useful,
% but WITHOUT ANY WARRANTY; without even the implied warranty of
% MERCHANTABILITY or FITNESS FOR A PARTICULAR PURPOSE.  See the
% GNU General Public License for more details.
%
% You should have received a copy of the GNU General Public License
% along with this program.  If not, see <http://www.gnu.org/licenses/>.
%



\section{Istruzione e Formazione}

\cventry{09/2013 - in corso}
{Informatica (Curriculum Sistemi e Reti)}
{Corso di Laurea Magistrale}
{Scuola di Scienze, Dipartimento di Informatica: Scienza e Ingegneria}
{Alma Mater Studiorum, Università di Bologna}
{Valutazione: \textbf{in corso} su 110.}

\cventry{09/2009 - 03/2013}
{Informatica}
{Corso di Laurea Triennale}
{Facoltà di Scienze MM.FF.NN., Dipartimento di Informatica}
{Alma Mater Studiorum, Università di Bologna}
{Valutazione: \textbf{108} su 110.}

\cventry{09/2004 - 07/2009}
{Liceo Scientifico Tecnologico}
{Scuola di Istruzione Superiore}
{Istituto Tecnico Commerciale e per Geometri Luigi Fantini}
{Vergato, Bologna}
{Valutazione: \textbf{76} su 100.}



\subsection{Riconoscimenti}

\cvitem{2007/2008}
{\textbf{Borsa di Studio} ``per essersi distinto con profitto nel corso
dell'anno scolastico''}

\cvitem{2005/2006}
{\textbf{Borsa di Studio} ``per essersi distinto con profitto nel corso
dell'anno scolastico''}

\cvitem{2004/2005}
{\textbf{Borsa di Studio} ``per essersi distinto con profitto nel corso
dell'anno scolastico''}



\subsection{Tesi Informatica Magistrale}

\cvitem{Titolo}
{\textbf{da definire}}

\cvitem{Relatore}
{Prof. Luciano Bononi}

\cvitem{Correlatore}
{Dott. Luca Bedogni}

\cvitem{Descrizione}
{Studio del comportamento di specifici \textit{algoritmi} nell'ambito dello
\textit{streaming audio/video adattivo su HTTP} facenti parte del
\textit{protocollo MPEG-DASH} tramite la loro implementazione su un
\textit{simulatore OMNET++}.}



\subsection{Tesi Informatica Triennale}

\cvitem{Titolo}
{\textbf{Un Framework per il Controllo e la Gestione Automatica dello
Spostamento di Sensori Mobili}}

\cvitem{Relatore}
{Prof. Luciano Bononi}

\cvitem{Correlatori}
{Prof. Marco Di Felice, Dott. Luca Bedogni}

\cvitem{Descrizione}
{Progettazione e sviluppo di un \textit{framework} composto da
un'\textit{applicazione Android} in grado di comandare il movimento di un
\textit{robot} collegato ad una \textit{scheda Arduino} tramite
\textit{interfaccia di comunicazione Bluetooth}.}

\subsection{Insegnamenti Universitari}

\cvitem{Magistrale}
{Oltre ad un approfondimento delle materie già affrontate durante il Corso di
Laurea Triennale come gli \textit{algoritmi}, studiati questa volta in ambiti
\textit{paralleli} e \textit{distribuiti} ed i \textit{database}, sono stati
seguiti insegnamenti specifici più incentrati sullo studio dei \textit{sistemi}
e delle \textit{reti}, dai \textit{sistemi complessi} a quelli
\textit{distribuiti}, dai \textit{sistemi middleware} a quelli \textit{mobili},
così come i \textit{sistemi e le reti wireless}. Sono poi stati seguiti Corsi
che hanno permesso di ampliare le conoscenze al di fuori dello specifico ambito
stabilito dal Curriculum scelto, quali \textit{intelligenza artificaile},
\textit{simulazione} e \textit{grafica}, quest'ultima anche da un punto di vista
dell'\textit{interazione persona-computer}, quindi volta alla progettazione di
interfacce grafiche che più si adattano alle specifiche esigenze dettate
dall'ambito d'uso.}

\cvitem{Triennale}
{Gli insegnamenti più significativi, oltre a vertere sull'apprendimento della
\textit{matematica} e della \textit{fisica}, si sono incentrati sullo studio
della \textit{logica} e della \textit{programmazione}, sia ad \textit{alto} che
a \textit{basso livello}, così come in \textit{ambito web} e \textit{mobile}.
Sono stati studiati gli \textit{algoritmi} e le \textit{strutture dati}
principali, i \textit{linguaggi di programmazione}, i \textit{database}, la
\textit{sicurezza} e come gestire lo sviluppo di un progetto nella sua
interezza, così come i principi dell'\textit{ingegneria del software} insegnano.
È stato studiato il funzionamento delle \textit{reti} e dei \textit{sistemi
operativi}. Infine, sono state studiate le \textit{teorie della calcolabilità} e
\textit{della complessità}.}
