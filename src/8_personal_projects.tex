%
% Curriculum Vitae is the personal CV of Giulio Biagini.
%
% Copyright (C) 2017 Giulio Biagini - giulio.biagini90@gmail.com
%
% This program is free software: you can redistribute it and/or modify
% it under the terms of the GNU General Public License as published by
% the Free Software Foundation, either version 3 of the License, or
% (at your option) any later version.
%
% This program is distributed in the hope that it will be useful,
% but WITHOUT ANY WARRANTY; without even the implied warranty of
% MERCHANTABILITY or FITNESS FOR A PARTICULAR PURPOSE.  See the
% GNU General Public License for more details.
%
% You should have received a copy of the GNU General Public License
% along with this program.  If not, see <http://www.gnu.org/licenses/>.
%



\section{Progetti}

\subsection{Personali}

\cventry{2017}
{Monitor YouTube Video}
{}
{}
{}
{Programma che può essere utilizzato sia da riga di comando che tramite
interfaccia grafica, il quale permette il monitoring circa le statistiche di un
video YouTube ad intervalli regolari.
\begin{itemize}
	\item Risorse: Java.
	\item GitHub: \url{https://github.com/GiulioBiagini/Monitor-YouTube-Video}
\end{itemize}}

\cventry{2016}
{NAS con RaspberryPi 2 e Samba}
{}
{}
{}
{Configurazione di un RaspberryPi 2 Model B in modo che agisca da NAS - Network
Attached Storage per la condivisione di memorie di massa su rete tramite la
suite di programmi per interoperabilità Windows/Linux Samba.
\begin{itemize}
	\item Risorse: RaspberryPi 2 Model B, Linux, Samba.
\end{itemize}}



\subsection{Universitari}

\cventry{2016}
{BattleSheep}
{Progetto di Sistemi Distribuiti}
{}
{}
{Programma distribuito che rievoca in modo divertente la battaglia navale.
\begin{itemize}
	\item Risorse: Java, RMI.
	\item GitHub: \url{https://github.com/DSBattleSheep/BattleSheep}
\end{itemize}}

\cventry{2015}
{PersonalPharm}
{Progetto di Interazione Persona-Computer}
{}
{}
{Analisi, progettazione, implementazione e testing dell'interfaccia della WebApp
PersonalPharm secondo il modello di processo circolare ISO 9241-210.
\begin{itemize}
	\item Risorse: modello di processo circolare ISO 9241-210, principi
	Human-Centered Design, approcci Task-Oriented e Goal-Oriented, metodologia
	del Card Sorting, Modello CAO=S, approccio Discount Usability Testing basato
	sulla metodologia del Thinking Aloud, valutazione basata sul modello Single
	Easy Question.
\end{itemize}}

\cventry{2015}
{PersonalPharm}
{Progetto di Complementi di Basi di Dati}
{}
{}
{WebApp che permette ad un ``caregiver'' di prendersi cura di alcuni
``pazienti''.\newline
Il caregiver registra un paziente, ne inserisce il piano terapeutico e l'app si
occupa di ricordare al paziente quando prendere le medicine. Una volta inseriti
i dati, il caregiver imposta il tablet e lo consegna ai pazienti.\newline
Il mio personale contributo è stata la realizzazione del software lato server.
\begin{itemize}
	\item Risorse (lato server): Python, Django (Python Web Framework).
	\item URL: \url{http://130.136.143.4}
\end{itemize}}

\cventry{2015}
{Collaborative Robbies}
{Progetto di Fisica dei Sistemi Complessi}
{}
{}
{Programma per lo studio circa la collaborazione di due agenti intelligenti che
agiscono in uno stesso ambiente al fine di massimizzare una misura di
prestazione comune.
\begin{itemize}
	\item Risorse: C, Algoritmi Genetici.
	\item GitHub: \url{https://github.com/DSBattleSheep/BattleSheep}
\end{itemize}}
